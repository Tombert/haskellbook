\chapter{Functions}

Let me get the shocking truth out right away: most languages don't have functions.  

Of course, nearly every language has a concept of a ``function", and might have a keyword designated for it, but, in a mathematical sense, they don't follow the major properties of functions.  

For example, in mathematics you may be familiar with the ``vertical line test" for checking whether a graph could represent a function. This simple procedure exists to show that for any single input value $ x $, in the function $f\left(x\right)$, you will always get the same value out.  

For example, let's look at the following function: 

$$ f\left( x \right) = x ^ 2 $$ 

If I supply the value $ 10 $ for $ x $, The result will \textit{always} be $ 100 $, no matter how many times I run it.  It will be $ 100 $ whether or not a certain file exists on a hard drive. It will be $100$ even if all of Google's servers break down.  It will be $100$ for always and forever simply because that is a property of functions. 

Now, let's think about the JavaScript function \inlinecode{Date.now()}.  It superficially looks like a function, and returns a value.  Most of the documentation calls it a function, and in the way most programmers think of it, it might be considered a function in some kind of abstract sense. 

But \inlinecode{Date.now()} doesn't pass the vertical line test. \inlinecode{Date.now()} takes no arguments, but returns a different result upon every execution.  \inlinecode{Date.now()} is dependent on the computer's system clock, which is likely dependent on a service like NTP or \inlinecode{time.nist.gov}. 

In functional programming, we distinguish between predictable, mathematical functions and functions dependent on external factors as \textit{pure} and \textit{impure}.  

Besides simple predictability, pure functions have several other advantages.  By using them, it becomes easier to push more reliance onto the type system of the language, which allows the compiler to diagnose many most potential errors before. 
